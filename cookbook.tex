\documentclass[a4paper,12pt]{article}
\usepackage[contents]{cuisine}
\usepackage{amsfonts}
\usepackage{amsmath}
\usepackage[polish]{babel}
\usepackage[utf8]{inputenc}
\usepackage[T1]{fontenc}
\usepackage[all]{nowidow}

\begin{document}

\tableofcontents



\newpage
\section{Ciasta}

\begin{recipe}{Ciasteczka imbirowe}{}{}
\Ingredient{3 szklanki mąki pszenno-razowej} 
\Ingredient{250g margaryny}
\Ingredient{1 łyżka miodu}
\Ingredient{100g orzechów/płatków migdałowych}
\Ingredient{1 płaska łyżka imbiru}
\Ingredient{1 płaska łyżeczka cynamonu}
\Ingredient{1 płaska łyżeczka proszku do pieczenia}
\Ingredient{\fr12 łyżeczki kukurmy}
Mąkę, margarynę, miód, orzechy i przyprawy zmieszać ze sobą krojąc nożem.
\Ingredient{200g cukru puderu}
\Ingredient{4 żółtka}
Dodać cukier oraz żółtka i zagnieść ciasto, po czym wstawić je do chłodnego miejsca na godzinę.
Wycisnąć ciasteczka przez foremkę w maszynce do mięsa i wyłożyć na blasze pokrytej papierem do pieczenia. Piec w temperaturze 160--180\0C przez 14 minut.
\end{recipe}

\begin{recipe}{Tarta z owocami}{}{}
\Ingredient{230g mąki pszennej}
\Ingredient{4 żółtka}
\Ingredient{\fr14 szklanki cukru}
\Ingredient{\fr12 łyżeczki proszku do pieczenia}
\Ingredient{125g margaryny}
\Ingredient{1 łyżka smalcu}
Mąkę, margarynę, smalec proszek do pieczenia zmieszać ze sobą krojąc nożem. 
Dodać cukier (\fr14 szklanki) oraz żółtka i rozwałkować ciasto. 
Wyłożyć na blasze pokrytej papierem do pieczenia. Piec w temperaturze 160\0C przez 20 minut.
\Ingredient{\fr34 szklanki cukru}
\Ingredient{4 białka}
\Ingredient{Truskawki/jagody}
Białka ubić na sztywną pianę dodając stopniowo cukier (\fr34 szklanki). Wyłożyć na upieczone ciasto i dodać owoce (truskawki/jagody). Piec przez 15--20 minut.
\end{recipe}

\begin{recipe}{Placek z zatapianymi owocami}{}{}
\Ingredient{3 jajka}
\Ingredient{1 szklanka cukru pudru}
\Ingredient{\fr12 margaryny}
\Ingredient{\fr32 szklanki mąki pszennej}
\Ingredient{2 łyżeczki proszku do pieczenia}
\Ingredient{4 duże jabłka}
Całe jajka i cukier puder utrzeć z margaryną. Dodać mąkę i proszek do pieczenia. Następnie wyłożyć ciasto na blachę i ułożyć owoce. Piec w temperaturze 160--180\0C przez 40 minut. Posypać cukrem pudrem.
\end{recipe}

\begin{recipe}{Ciasto drożdżowe z rabarbarem i kruszonką}{}{}
\freeform%
\begin{center}
    \textit{Zamiast rabarbaru można dodać jabłka, brzoskwienie lub borówki.}
\end{center}
\Ingredient{50 g świeżych drożdży}
\Ingredient{1 szklanka mleka}
\Ingredient{2 łyżki oleju}
\Ingredient{3 łyżki cukru}
\Ingredient{1 cukier waniliowy}
\Ingredient{3 szklanki mąki pszennej}
\Ingredient{2 żółtka}
\Ingredient{\fr12 łyżeczki soli}
\Ingredient{2 krótkie łodygi rabarbaru}
W misce zasypać drożdże cukrem i zostawić do rozpuszczenia.
Następnie wbić żółtka, wsypać cukier waniliowy i sól.
Wszystkie składniki wymieszać łopatką.
Wlać ciepłe mleko z olejem oraz wsypać mąkę.
Wymieszać ciasto dokładnie łopatką aż będzie odstawało od brzegów miski, posypać je po wierzchu mąką, uformować w misce kulę.
\freeform{}%
Do większej miski lub zlewu nalać dobrze ciepłej wody, ale nie gorącej. Wstawić miskę do wody, przykryć ściereczką i zostawić na pół godziny. \\

Kiedy ilość ciasta podwoi się wyjąć na stolnicę i lekko zarobić. Przełożyć ciasto do przygotowanej blaszki wyłożónej papierem do pieczenia, zostawić do wyrośnięcia. \\

Rabarbar (lub owoce) opłukać, obrać i pokroić w nieduże kawałki. Posmarować ciasto rozbitym białkiem i rozłożyć na nim owoce. 
\Ingredient{125 g margaryny}
\Ingredient{125 g mąki}
\Ingredient{125 g cukru}
\Ingredient{60 g krojonych orzechów}
Składniki na kruszonkę posiekać, szybko zagnieść. Drobne kawałki rozłożyć na wierzchu ciasta. Piec w temperaturze 180\0C przez 40 minut.
\end{recipe}

\begin{recipe}{Sernik bez sera}{}{}
\freeform{}%
\Ingredient{3 szklanki mągki}
\Ingredient{1 szklanka cukru pudru}
\Ingredient{1 kostka margaryny}
\Ingredient{6 żółtek}
\Ingredient{1 płaska łyżka proszku do pieczenia}
Składniki wymieszać, ciasto podzielić na pół. Jedną część zamrozić, drugą schłodzić. 
\Ingredient{4 śmietany 18\%}
\Ingredient{6 białek}
\Ingredient{\fr32 szklanki cukru}
\Ingredient{2 budynie śmietankowe}
Ubić białka, następnie dodać cukier i wlewać po 1 śmietanie. Cały czas mieszając dodać 2 budynie.
Na blachę rozłożyć schłodzone ciasto, wylać masę i na koniec zetrzeć zmrożone ciasto na wierzch.
Piec w temperaturze 170\0C przez 40 minut.
\end{recipe}


\newpage
\section{Pieczywa}


\begin{recipe}{Grahamki idealne}{}{}
\Ingredient{1 szklanka letniego mleka lub maślanki}
\Ingredient{1 szklanka letniej wody}
\Ingredient{1 łyżka miodu}
\Ingredient{2 łyżeczki soli}
\Ingredient{5 szklanek mąki grahama* (mąka pszenna typ 1850)}

\Ingredient{2 szklanki mąki pszennej zwykłej}
\Ingredient{60 ml oleju (4 łyżki)}
\Ingredient{40 g świeżych drożdży lub 14 g suchych drożdży** (2 paczuszki)}

Świeże drożdże rozetrzeć z miodem, dodać połowę mleka i wody, wsypać ze 2--3 łyżki mąki i poczekać aż ``ruszą'',
następnie dodać pozostałe mleko i wodę wymieszane z olejem. Jeśli używamy suchych drożdży to  wymieszać je z mąkami.

Dosypać obie mąki --- od razu 5 szklanek mąki grahama i 1.5 szklanki mąki pszennej zwykłej, pozostałe \fr12 szklanki dosypywać powoli w zależności od jej wilgotności. Wyrabiać około 10 minut, ciasto
może pozostać \underline{lekko} klejące.

\freeform{}%
Przykryć ściereczką i pozostawić przykryte w ciepłym miejscu aż do podwojenia obiętości (około 1--1.5 godziny w zależności od temperatury, można wstawić do piekarnika o temperaturze 35\0C i
wystarczy 40 minut).\\

Wyrośnięte ciasto podzielić na około 70--75 gramowe części (wychodzi 20 bułeczek), następnie oprószając dłonie mąką lub smarując olejem uformować podłużne bułeczki, ułożyć na
blaszce wyłożonej papierem do pieczenia (można ten papier używać kilkakrotnie) przykryć, pozostawić w cieple do napuszenia i podwojenia objętości (około 30--40 minut).\\

Przed pieczeniem posmarować ciepłą wodą ``zabieloną'' mąką i posypać czym się chce. Woda z mąką ``przytrzymuje'' posypkę. Wstawić do piekarnika nagrzanego do 250\0C, ścianki i dno piekarnika spryskać wodą, po 10 minutach
temperaturę zmniejszyć do 190\0C i piec kolejne 10 minut. Po upieczeniu wyłożyć na kratkę i spryskiwać zimną wodą z rozpylacza. Po ostudzeniu nadmiar bułek zamrozić.

\end{recipe}

\newpage
\begin{recipe}{Chleb na zakwasie}{}{}
\freeform%
\begin{center}
\textit{Zakwas można trzymać w lodówce 8 dni.}
\end{center}

\Ingredient{1 litr wody (ciepłej) + 1.5 łyżki soli}
\Ingredient{700 g mąki pszennej}
\Ingredient{300 g mąki żytniej}
\Ingredient{\fr14 szklanki słonecznika}
\Ingredient{\fr14 szklanki dyni}
\Ingredient{\fr12 szklanki otrąb pszennych}
\Ingredient{\fr12 szklanki płatków owsianych}
\Ingredient{\fr12 szklanki siemienia lnianego}

Na 12 godzin przed pieczeniem wyciągnąć zakwas z lodówki i wymieszać w misce z 100g mąki żytniej i 100g wody letniej i odstawić do wyrośnięcia.

\freeform%
Po 12 godzinach sporządzić ciasto.

Do zakwasu dodać wody, wsypać wszystkie składniki --- miksować przez 5--10 min.
Rozłożyć ciasto do wcześniej przygotowanych dwóch podłóżnych foremek wysmarowanych olejem i posypanych mąką. Wierzch ciasta posmarować wodą i
ewentualnie posypać makiem lub płatkami owsianymi.

Foremki wstawić do zimnego piekarnika i pozostawić na minimum 8 godzin. Po tym czasie włączyć piekarnik, gdy temperatura osiągnie 180\0C piec przez 1 godzinę.

Po upieczeniu chleby wyłożyć i przykryć ściereczką do wystygnięcia.

\end{recipe}
\newpage

\begin{recipe}{Biały chleb na zakwasie}{2 bochenki}{}
\Ingredient{800 g mąki pszennej}
\Ingredient{10 g soli}
\Ingredient{460 ml wody}
\Ingredient{320 g zakwasu}

W misce zmieszać mąkę z solą, następnie dodać wodę, zakwas i połączyć ze sobą
wszystkie składniki. Zagnieść ciasto przez 10 minut tak, aby było miękkie i
elastyczne (nie przejmować się, jeśli ciasto jest wilgotne lub klejące, nie
dodawać mąki).  
\freeform%
Odłożyć ciasto spowrotem do miski i przykryć folią spożywczą na
4 godziny w temperaturze pokojowej. Po 4 godzinach wyjąć ciasto i zagnieść
ponownie, aby usunąć z niego powietrze i uformować kulę.
Koszyk posypać mąką i włożyć do niego ciasto tak aby spoina była skierowana do
góry. Przykryć luźno sciereczką i pozostawić na 3--3\fr{1}{2} h. Piekarnik
ustawić na 230\0C (210\0C z termoobiegiem) i wstawić płytką foremkę
na spód piekarnika, aby się również zagrzała. Ostrożnie wyłożyć
ciasto z koszyka tak, aby spoina była skierowana na formę do pieczenia chleba. Używając ostrego
noża przeciąć powierzchnię ciasta nawet na głębokość \fr{1}{2} cm. Zagotować
wodę i wylać na wcześniej zagrzaną blachę --- w trakcie pieczenia
będzie się tworzyć w piekarniku para. Foremkę z ciastem wstawić do piekarnika i
piec przez 30--35 minut.

\end{recipe}

\newpage
\begin{recipe}{Bułka wrocławska}{2 bułki}{}
\Ingredient{600 g mąki pszennej}
\Ingredient{160 ml mleka}
\Ingredient{200--220 ml wody}
\Ingredient{15 g świeżych drożdży}
\Ingredient{2 łyżeczki cukru}
\Ingredient{12 g soli (1.5 łyżeczki)}
\Ingredient{2 łyżki oleju}
\Ingredient{1 łyżkę otrębów pszennych (na ciasto tostowe)}

Drożdże pokruszyć i zasypać 2 łyżeczkami cukru. Kiedy się roztopią dodać do ciepłego (nie gorącego) mleka i odstawić na chwilę. Wsypać do miski mągkę, na środku zrobić dołek, wlać drożdże i dodać
pozostałe składniki.
\freeform%

Zagnieść ciasto --- będzie miękkie i niezbyt klejące. Ręcznie należy je wyrabiać ok. 15 minut (mikserem ok. 7 minut). Wyjąć na oprószony mąką blat, uformować z niego kulę, delikatnie posmarować
olejem i przełożyć do miski. Przykryć folią i odstawić do wyrastania na 1--1.5 godziny. \\

Wyjąć ciasto, lekko je spłaszczyć i podzielić na 2 części. Z każdej uformować podłużną, niezbyt cienką bułkę. Ułożyć na blasze wyłożonej papierem, przykryć ściereczką i zostawić do wyrastania na
45--60 minut.\\

Piekarnik nagrzać do 200\0C. Na dno wsypać \fr12--1 szklankę kostek lodu. Kiedy się roztopią naciąć bułki ostrym nożem lub żyletką, powierzchnię posmarować delikatnie wodą i wstawić do piekarnika.
Po 10 minutach obniżyć temperaturę do 180\0C i piec kolejne 20--30 minut.\\

Jeśli wierzch bułek zbyt szybko się rumieni przykryć je folia aluminiową. Po upieczeniu posmarować jeszcze raz wodą i wstawić do wyłączonego, ale gorącego piekarnika na 3 minuty. Ostudzić na
kuchennej kratce.

\end{recipe}


\newpage
\begin{recipe}{Bułki drożdżowe}{18 bułek}{}
\Ingredient{4 szklanki mąki pszenna}
\Ingredient{2 szklanki mąki żytniej}
\Ingredient{100 g drożdży}
\Ingredient{1 płaska łyżka soli}
\Ingredient{3 łyżki oleju}
\Ingredient{1.5 szklanki wody (ciepłej)}
\Ingredient{\fr12 szkalnki mleka}

W misce rozpuścić drożdże dodając 1 łyżeczkę cukru, dodać szklankę ciepłej wody i 2 szklanki mąki pszennej i odstawić do wyrośnięcia. Następnie dodać sól, resztę mąki oraz olej.

\freeform%
Wszystko dobrze wyrobić mikserem (ok. 10 minut) i znów pozostawić do wyrośnięcia w ciepłym miejscu (można wstawić do piekarnika na 35\0C). Gdy urośnie podzielić ciasto na bułki, rozłożyć na wysmarowanej blasze lub papierze
do pieczenia (papier do pieczenia można użyć kilkakrotnie). Piec w temperaturze 200\0C przez 30 minut. 

\end{recipe}

\begin{recipe}{Tortilla}{}{}
\Ingredient{2 szklanki mąki pszennej}
\Ingredient{160 ml gorącej wody}
\Ingredient{2 łyżki oleju}
\Ingredient{szczypta soli}

W misce zmieszać mąkę z solą i zalać gorącą wodą (2--3 minuty po zagotowaniu).
Dodać olej i mieszać składniki drewnianą łyżką, a następnie zagnieść rękoma przez 5 minut.
Wyrobione ciasto przykryć ściereczką i odstawić na kilkanaście minut.

\freeform%
Ciasto podzielić na 8 części. Z każdej z nich uformować kulkę, rozpłaszczyć, a następnie wałkować jak na koło o średnicy o 1--2 cm większe niż średnica dna patelni. 
Jeżeli jest taka potrzeba można w trakcie wałkowania minimalnie podsypać mąką. 
 
Smażyć na suchej patelni z powłoką non-stick aż do zarumienienia (powinna się zwijać bez kruszenia).

\end{recipe}

\newpage
\section{Pizze}

\begin{recipe}{Pizza na zakwasie}{}{}
\Ingredient{500 g mąki pszennej}
\Ingredient{50 g mąki z przenicy durum}
\Ingredient{10 g soli}
\Ingredient{200 g zakwasu}
\Ingredient{\fr{1}{4} g drożdży}
\Ingredient{325 ml wody}
\Ingredient{50 ml oliwy z oliwek}

W misce zmieszać mąki i sól. Rozpuścić drożdże w wodzie i dodać do mąki
razem z oliwą z oliwek i zakwasem. Wszystkie składniki wymieszać (ok. 30 s).
Ciasto przykryć folią spożywczą i wstawić do lodówki na 24 godziny.
\freeform%
Następnego dnia rozgnieść ciasto i podzielić na 6 części (ok. 200 g każda).
Wstawić ciasto do posypanej mąką foremki do pieczenia, przykryć ściereczką i
odstawić na 60--90 minut. Piekarnik ustawić na najwyższą temperaturę i wstawić
blachę do góry nogami do zagrzania. Uformować ciasto w koło i przenieść na
deskę pokrytą mąką. Deksa posłuży do przełożenia spodu pizzy do
piekarnika. Pokryć wierzch pizzy sosem pomidorowym i nałożyć wybrane składniki.
Tak przygotowaną pizzę wstawić do piekarnika na 4\fr{1}{2} do 5 minut.

\end{recipe}

\newpage
\section{Potrawy drobiowe}

\begin{recipe}{Kurczak słodko-kwaśny z woka}{1 porcja}{}
\Ingredient{200g piersi z kurczaka}
\Ingredient{1 łyżka ciemnego sosu sojowy}
Marynata --- do miski dodać ciemny sos sojowy i pokrojoną w kostkę pierś z
kurczaka.

\Ingredient{\fr{1}{3} marchewki}
\Ingredient{1 różyczka brokuła}
\Ingredient{2 ząbki czosnku}
\Ingredient{imbir (kostka 1$\times$2 cm)}
Marchewkę i brokuły pokroić w zapałki. Imbir oraz czosnek obrać i pokroić w małą
kostkę. 
\Ingredient{Pomarańcza}
\Ingredient{\fr{1}{2} limonki}
\Ingredient{1 łyżeczka octu ryżowego}
\Ingredient{imbir (kostka 1$\times$2 cm)}
\Ingredient{2 łyżeczki ciemnego sosu sojowego}
\Ingredient{25g miodu}
\Ingredient{1 łyżeczka skrobii ziemniaczanej}
Sos --- do miski dodać sok z pomarańczy, sok z limonki, imbir, ciemny sos
sojowy, podgrzany miód oraz skrobię ziemniaczaną i wymieszać.
\Ingredient{ryżu}
Ryż gotować przez 35 minut w osolonej wodzie.
\freeform%
Nagrzać woka aż do dymienia, następnie dodać ok 20 ml oleju. Przez 2 minuty
smażyć mięso, następnie odłożyć na talerz. Ponownie dodać oleju i smażyć
czosnek i imbir przez 1--2 minuty. Następnie dodać brokuły, po chwili
marchewkę. Po podsmażeniu dodać mięso, a następnie sos i mieszać aż do
zagęszczenia.

\end{recipe}

\newpage

\begin{recipe}{Kurczak Kung Pao z makaronem}{3 porcje}{}
\freeform%
Surówka z kapusty:
\Ingredient{\fr{1}{2} kg białej kapusty}
\Ingredient{125g marchewki}
\Ingredient{25 ml wody}
\Ingredient{25 ml oleju ryżowego}
\Ingredient{30g cukru}
\Ingredient{40 ml octu ryżowego}
\Ingredient{sól}

Kapustę i marchewkę poszatkować na drobne słupki. Zalewę podgrzać aż do ropuszczenia cukru i dodać do surówki. Wstawić na 24h do lodówki.

\freeform%
Makaron:
\Ingredient{25g białej kapusty}
\Ingredient{25g marchewki}
\Ingredient{5g grzybów mun}
\Ingredient{240g makaronu chińskiego}
\Ingredient{25g pory}
\Ingredient{15 ml oleju ryżowego}
\Ingredient{10 ml oleju sezamowego}
\Ingredient{100 ml wody}
\Ingredient{1 łyżeczka miodu}
\Ingredient{3 łyżki jasnego sosu sojowego}
\Ingredient{1 łyżeczka kurkumy}
\Ingredient{1 płaska łyżeczka zmielonego kminu rzymskiego}
\Ingredient{cynamon}
Kapustę i marchewkę poszatkować (słupki). Grzyby mun zalać wrzątkiem. Zagotować makaron przez 5--6 minut w osolonej wodzie z dodatkiem \fr{1}{2} łyżeczki kurkumy.
Por pokroić w kostkę, grzyby mun w niewielkie paski. Na rozgrzaną patelnię dodać olej ryżowy i dodajemy warzywa z grzybami.
Po 2 minutach dodajemy makaron oraz resztę przypraw (\fr{1}{2} łyżeczki kurkumy, łyżeczka kminu, szczypta cynamonu). Na niewielkim płomieniu przemieszać i dodać olej sezamowy, miód rozpuszczony w wodzie oraz sos sojowy.

\newpage

\freeform%
Kurczak Kung Pao:
\Ingredient{450g piersi z kurczaka}
\Ingredient{1 papryka (zielona + czerwona)}
\Ingredient{45g orzeszków ziemnych arachidowych}
\Ingredient{1 cebula dymka}
\Ingredient{3 papryczki chilli}
\Ingredient{4 ząbki czosnku}
\Ingredient{imbir (kostka 3$\times$3 cm)}
\Ingredient{30 ml oleju}
\Ingredient{4 łyżki jasnego sosu sojowego}
\Ingredient{4 łyżki ciemnego sosu sojowego}
\Ingredient{4 łyżek octu balsamicznego}
\Ingredient{1 łyżeczka mąki ziemniaczanej}

Kurczaka pokroić w kostkę i wymieszać z 1 łyżką sosu (sosy sojowe z octem balsamicznym). Czosnek i imbir drobno posiekać, szczypior pokroić na kawałki ok 3 cm, paprykę chilli pokroić na kawałki ok 2 cm. 
Paprykę słodką pokroić w trójkąty. Do sosu dodać 250 ml wody i skrobię. Patelnię zagrzać i smażć orzeszki na złoty kolor. 15ml oleju wlać na patelnię i usmażyć mięso. Po usmażeniu odkłożyć ndo miski, a następnie smażyć imbir, czosnek, paprykę chilli i połowę cebulki dymki. Następnie dodać paprykę, i po kilku minutach kurczaka z orzeszkami. Dodać sos i mocno podgrzać aż do zagotowania.

\end{recipe}


\newpage

\section{Potrawy wołowe}

\begin{recipe}{Wołowina po pekińsku z makaronem na ostro z woka}{1 porcja}{}
\Ingredient{150 g wołowiny krzyżowej}
\Ingredient{2 łyżki ciemnego sosu sojowy}
\Ingredient{2 łyżki octu ryżowy}
Marynata --- w misce zmieszać ciemny sos sojowy z octem ryżowym. Dodać
pokrojone mięso wołowe w plastry (5--7mm grubości) do marynaty.

\Ingredient{1 cebula}
\Ingredient{\fr{1}{2} czerwonej papryki}
\Ingredient{2 ząbki czosnku}
\Ingredient{imbir (wielkości ząbka czosnku)}
\Ingredient{1 papryka chili}
Cebulę i czerwoną paprykę pokroić w grubą kostkę. Czosnek i imbir obrać i
pokroić w drobną kostkę. Paprykę chili pokroić w drobne plasterki.

\Ingredient{2 łyżki jasnego sosu sojowego}
\Ingredient{1 łyżka ciemnego sosu sojowy}
Sos --- do miski wlać 50 ml wody, jasnego sosu sojowego i ciemnego sosu sojowego. 
\Ingredient{55g makaronu}
Makaron gotować przez 7 minut w osolonej wodzie.
\freeform%
Nagrzać woka aż do dymienia, następnie dodać ok 20 ml oleju. Przez 2 minuty
smażyć mięso, następnie odłożyć na talerz. Ponownie dodać oleju i smażyć
czosnek, imbir, paprykę chili przez 1--2 minuty. Następnie dodać cebulę i
czerwoną paprykę. Po podsmażeniu cebuli dodać sos i zagotować przez minutę. Do
woka dodać makaron oraz mięso.

\end{recipe}

\newpage

\section{Owoce morza}

\begin{recipe}{Krewetki z makaronem}{2 porcje}{}
\Ingredient{500 g krewetek (300g obranych)}
Krewetki obrać (odrerwać głowę, a następnie zdjąć pancerzyk), naciąć grzbiet i wyciągnąć jelita. Następnie mięso umyć w zimnej wodzie i wysuszyć ręcznikiem papierowym.
\Ingredient{4 ząbki czosnku}
\Ingredient{1 papryka chili}
\Ingredient{6 suszonych pomidorów}
\Ingredient{2 garści pietruszki}
Czosnek obrać i pokroić w cienkie plasterki. Paprykę chili pokroić w cienkie paseczki (można usunąć pestki aby potrawa nie wyszła ostra).
Pomidory pokroić w podłużne paski a pietruszkę posiekać.
\Ingredient{120 g makaronu spaghetti} 
\Ingredient{2 garści szpinaku}
\Ingredient{20--25 g masła}
\Ingredient{30 g oliwy}
\Ingredient{2 łyżki octu jabłkowego}
Na patelnię dodać oliwę, czosnek oraz paprykę. Smażyć na niskim ogniu, aż czosnek się zarumieni.
Następnie dodać krewetki, i smażyć z obu stron aż zrobią się pomarańczowe i delikatnie przyrumienione.
\freeform%
W trakcie smażenia krewetek makaron wrzucić do gotującej, osolonej wody i gotować przez ok 8 minut. Do patelni dodać masło, następnie szpinak oraz suszone pomidory. Jak szpinak zwiotczeje do patelni dodajemy ocet oraz odrobinę wody z gotowanego makaronu. Dodajemy odrobinę pietruszki, a po ugotowaniu makaronu mieszamy wszystkie składniki na patelni przez 2 minuty. Na koniec posypujemy resztą pietruszki oraz świeżo zmielonym pieprzem kolorowym.

\end{recipe}

\newpage
\section{Sałatki}

\begin{recipe}{Sałatka jarzynowa}{4 porcje}{}
\Ingredient{2 średnie ziemniaki}
\Ingredient{2 średnie marchewki}
\Ingredient{1 pietruszka}
\Ingredient{\fr{1}{4} selera}
\Ingredient{1 jabłko}
\Ingredient{1 ogórek kiszony}
\Ingredient{1 mała cebula}
\Ingredient{1 puszka groszku}
\Ingredient{1 puszka kukurydzy}
\Ingredient{2 łyżki majonezu}

Ziemniaki gotować przez 20 minut. Marchewkę, seler i pietruszkę obrać i gotować przez 17 minut.
Składniki pokroić w kostkę, dodać majonez, sól i pieprz. Wszystko wymieszać.

\end{recipe}

\newpage
\section{Wędzenie}

\begin{recipe}{Szynka wieprzowa}{}{}

\freeform%
Wywar z przypraw:
\Ingredient{250 ml wody}
\Ingredient{20g peklosoli}
\Ingredient{1.5 g liścia laurowego}
\Ingredient{4g ziela angielskiego}
\Ingredient{5g ziaren pieprzu czarnego}

Składniki zagotować przez 15 minut. Po tym czasie ostudzić, a następnie odcedzić przez sito i przez gazę.

\freeform%
Peklowanie:
\Ingredient{Szynka wieprzowa}
\Ingredient{1 litr wody}
\Ingredient{100g peklosoli}


Na każdy 1kg mięsa przypada 0.4--0.7 litra zalewy.
Do wody wsypać peklosól i wymieszać. Na 1 kilogram szynki wstrzyknąć 100ml wywaru z przypraw.
\freeform%
Następnie włożyć mięso do roztworu peklosoli, dolać resztę wywaru i wstawić do lodówki na 5 dni (temperatura 6 stopni). 
Raz dziennie przemieszać szynki, aby równomiernie się peklowały.
Po 5 dniach włożyć szynki do siatki.
Zasznurowane mięso należy pozostawić do ociekania na 10--14 godzin w temperaturze 14--18\0C. \\

Wędzienie:

Rozgrzać wędzarnie do temperatury 50--60 \0C, aby pozbyć się wilgoci.
Na samym żarze, aby nie było wiele dymu suszyć mięso przez 30 minut.
Następnie rozpocząć wędzenie na gorąco w temperaturze 60--65\0C przez 2--3 godziny. Mięso należy obracać i zmieniać ich ułożenie, aby równomiernie się uwędziły. \\

Parzenie:
Wodę w garnku należy rozgrzać do 75--80\0C. Szynkę parzymy, aż osiągnie 68\0C. 
Szynkę wyciągnąć i studzić w lodowatej wodzie na 10 minut. Po tym czasie wymienić wodę i chłodzić przez kolejne 10 minut. Kontynuować proces chłodzenia (na dworze/strychu) i wstawić mięso do lodówki na 24h.
\end{recipe}

\end{document}

